\documentclass{iitthesis}


% Document Options:
%
% Note if you want to save paper when printing drafts,
% replace the above line by
%
%   \documentclass[draft]{iitthesis}
%
% See Help file for more about options.

\usepackage{graphicx}    % This package is used for Figures

\begin{document}

%%% Declarations for Title Page %%%
\title{Acceptance of Code Contributors on GitHub}
\author{Matthew Heston}
\degree{Master of Science}
\dept{Information Architecture}
\date{May 2014}
% \copyrightnoticetrue      % crate copyright page or not
%\coadvisortrue           % add co-advisor. activate it by removing % symbol to add co-advisor
\maketitle                % create title and copyright pages


\prelimpages         % Settings of preliminary pages are done with \prelimpages command


%%%  Acknowledgement %%%
\begin{acknowledgement}     % acknowledgement environment, this is optional
\par  This dissertation could not have been written without Dr. X
who not only served as my supervisor but also encouraged and
challenged me throughout my academic program. He and the other
faculty members, Dr. Y and Dr. Z, guided me through the
dissertation process, never accepting less than my best efforts. I
thank them all.\\ \\ (Don't copy this sample text. Write your own
acknowledgement.)
% or \input{acknowledgement.tex} % you need a separate acknowledgement.tex file to include it.
\end{acknowledgement}


% Table of Contents
\tableofcontents
\clearpage

% List of Tables
\listoftables

\clearpage

%List of Figures
\listoffigures

\clearpage

%List of Symbols(optional)

% \listofsymbols
%  \SymbolDefinition{$\beta$}{probability of non-detecting bad data}
%  \SymbolDefinition{$\delta$}{Transition Coefficient Constant for the Design of Linear-Phase FIR Filters}
%  \SymbolDefinition{$\zeta$}{Reflection Coefficient Parameter}


%  \clearpage



%%% Abstract %%%
\begin{abstract}           % abstract environment, this is optional
\par Your Abstract goes here!
% or \input{abstract.tex}  %you need a separate abstract.tex file to include it.
\end{abstract}


\textpages     % Settings of text-pages are done with \textpages command

% Chapters are created with \Chapter{title} command
\Chapter{INTRODUCTION}

GitHub is a social website that open source software developers use to host
their software projects and to browse other developers' projects. It includes
many features that are present on social networking sites, such as the ability
to follow other users and leave comments on projects. GitHub provides a wealth
of data for studying computer supported cooperative work, as it is a
centralized location where many different tasks take place. For example, users
can create bug reports, submit fixes, and engage in discussions about new
features all on one website.  In this study, we use statistical and machine
learning methods on data from GitHub repositories to explore how different
factors may affect the acceptance of code changes from first time contributors.

The rest of this paper is organized as follows. The rest of this chapter
provides a literature review on the subject, establishing the importance of
studying open source software development, situating it within a context of
virtual work and computer mediated communication, and reviewing a theoretical
basis we use to inform our empirical methods. Chapter~\ref{chap:methods}
describes our data collection and data analysis methods. The results of our
experiments are discussed in Chapter~\ref{chap:results}. We conclude in
Chapter~\ref{chap:conclusion} and discuss opportunities for future work.


\Section{Related Work} \label{sec:relatedwork}

\Subsection{FLOSS Research}
Research in the development of free/libre open source software (FLOSS) has
grown tremendously in the last several years. Crowston et
al.~\cite{crowston_free/libre_2008} note the importance of understanding FLOSS
development as it becomes a major social movement with many volunteers
contributing to projects, and many FLOSS projects becoming integral parts of the
infrastructure of modern societ as it becomes a major social movement with many
volunteers contributing to projects, and many FLOSS projects becoming
integral parts of the infrastructure of modern society. Other studies have
emphasized the role that FLOSS research can play in improving current existing
research of software engineering, particularly as the importance of
understanding large scale software systems in science and insustry
increases ~\cite{scacchi_free/open_2007}.

Existing research approaches FLOSS from many different angles, including
motivation of open source
developers~\cite{fang_understanding_2009},
~\cite{lakhani_why_2003},~\cite{shah_motivation_2006}; governance of open souce
projects~\cite{hippel_open_2003},~\cite{omahony_guarding_2003},~\cite{omahoney_governance_2007};
and knowledge sharing within FLOSS
communities~\cite{endres_tacit_2007},~\cite{hemetsberger_collective_2009},~\cite{sowe_understanding_2008}.
Our study focuses on the behavior of first time contributors to FLOSS projects
and community response to their contributions. We build on previous studies that
describe the social processes of community
joining~\cite{ducheneaut_socialization_2005},~\cite{huang_mining_2005}, ~\cite{von_krogh_community_2003}. Given the
distributed nature of FLOSS development, our findings contribute to current
descriptions of virtual work and distributed teams.
%TODO new members

Previous studies have used version control histories to verify learning
processes of new members in FLOSS projects~\cite{huang_mining_2005}. GitHub,
however, has not been extensively studied as it is a relatively new social
platform. Dabbish et al.~\cite{dabbish_social_2012} study how GitHub as a social
application provides transparency and how that transparency affects
collaboration and learning. McDonald and
Goggins~\cite{mcdonald_performance_2013} study how different communities on
GitHub measure success. Choi et al.~\cite{choi_herding_2013} study a sample of
GitHub based projects to contribute to theories of developer coordination. In
all these cases, the social features of GitHub, e.g. the ability to follow other
users and view information about them, provide new ways to study social behavior
in FLOSS projects. Our study investigates members' participation in group
discussions on the site. While previous studies have tried to combine data from
mailing lists and version control~\cite{ducheneaut_socialization_2005}, GitHub
provides a centralized location to study communities in which discussion and
code contribution all occur in one place. At least with regards to user support,
recent research suggests that developers may be moving away from mailing lists
to social Q\&A sites to respond to user requests for
help~\cite{vasilescu_how_2014}. By focusing on GitHub data, we contribute to
understanding developer behavior on this new social platform.


\Subsection{Communities of Practice} \label{sec:communities}
Our study focuses on the behavior of new code contributors and community
response to their contributions. We use the theoretical framework of
\textit{legitimate peripheral participation} (LPP) ~\cite{lave_situated_1991} in
our exploration community joining. LPP describes a process of learning in
communities of practice in which newcomers join a community by participating in
peripheral tasks and forming relationships to move towards the center of the
community. Several studies of FLOSS development have used the LPP framework.
Huang and Liu~\cite{huang_mining_2005} mine version control history to construct
developer networks and identify core and peripheral community members.
Ducheneaut~\cite{ducheneaut_socialization_2005} finds a pattern that resembles
LPP in his study of contributors to the Python project. Ye and
Kishida~\cite{ye_toward_2003} use LPP to ground their theory of motivation in
open source communities. This concept has been explored in other studies of
computer mediated communication.  In their study on members of Wikipedia,
~\cite{bryant_becoming_2005} note that members initially become involved through
peripheral activities. These are simple and low risk activities members can take
part in to learn more about the community before trying to become major
contributors. Similarly, ~\cite{von_krogh_community_2003} from observing open
source communities generate the construct of a \textit{joining script}, where
each project has a set of tasks for new developers to go through before being
accepted into the community.

Our study seeks to which factors contribute to the acceptance of code
contributions of first time contributors. We use LPP as a theoretical framework
to ground our experiments.

\Chapter{METHODS} \label{chap:methods}

\Section{Terminology} \label{sec:terms}

A software project on the website is referred to as a \textit{repository}. Any
user on GitHub can \textit{star} a repository. Users star repositories to be
able to easily navigate to it and to receieve updates on activity from the
repository. If a developer wants to contribute to another one of developer's
repositories, he can \textit{fork} the repository, which creates a copy of the
project for him to work on. As the developer makes changes to this code, he
\textit{commits} his changes. A \textit{commit} is a snapshot of the code at a
certain point in time. When the developer is finished, he can submit a
\textit{pull request} to the owner of the project. All pull requests for a
project are viewable on GitHub, and any user of the site can comment on them. A
pull request can have a status of open or closed. A status of open indicates
that that owner of the repository has not made a decision about whether or not
to include the changes. If the owner of a repository wants to incorporate the
changes the developer made, he can \textit{merge} them into the repository. A
pull request can be closed without being merged, which means that the changes
the developer made were not accepted.

\Section{Data Collection} \label{sec:datacollection}

Data was collected using the GitHub API.\footnote{http://developer.github.com/}
We used a collection of node.js scripts to collect data from the API to store in
a MySQL database.\footnote{These scripts are available at
http://www.github.com/matthewheston/gh-collector.} In selecting which
repositories to use for our analysis, we started with the top 100 most starred
repositories on GitHub. We started with this list with the assumption that they
were popular repositories that would be maintained by an active community. From
these 100, we manually filtered out certain projects that we expected would
follow different development patterns than a typical programming project, for
example, collections of configuration files for text editors and shells,
collections of icons, etc. We also excluded repositories that were used
primarily for demonstration or documentation purposes, such as sample web
applications to demonstrate use of a certain web framework. After filtering our
intial list of 100, 45 repositories remained in our data set for analysis. We
only consider pull requests with a status of closed. This resulted in
approximately 44,400 pull requests. We further filtered this data by select only
the first pull request a user submitted to a repository, leaving 13,383 pull
requests. The distribution of these pull requests across repositories ranges
from 10 to 1,489, with a median of 210. To find merged pull requests, we first
filter all pull requests that are marked as merged by the GitHub API, meaning
that the project maintainers used GitHub's merge feature to accept the pull
request. In some repositories, project maintainers use a different workflow when
accepting pull requests, wherein the code changes are accepted, but it is not
reflected as merged on GitHub. In most of these cases, there is a standard way
of reflecting this in the commit comments, so we use some naive heuristics for
identifying these requests by searching commit comments for certain text
patterns. For example, in many projects, the project maintainer will manually
add the commits from the pull request, and create a new commit with a commit
message that follows the pattern "Closes {number}" where {number} is the pull
request ID on GitHub. Finding merged pull requests using both the status from
the GitHub API as well as these text patterns results in finding 5,239, or
39.1\% of first pull requests being merged.

\Section{Data Analysis} \label{sec:data_analysis}

\Chapter{RESULTS} \label{chap:results}

\Section{Communities of Practice} \label{sec:communities}

In this section, we explore both how a user engages with the community before
submitting their pull request, as well as try to measure the community response
to the pull request.  We consider the main peripheral activity a user can
participate in on GitHub is commenting on other pull requests. For each first
pull request in our dataset, we count the number of other pull requests the user
commented on before submitting. Based on the these previous findings, we would
expect to see a positive correlation between this number and the likelihood that
a pull request is recieved. We also count the number of comments that each first
pull request recieves, with the assumption that this variable can be used to
measure the amount of community interest in a given pull request. We plot these
variables in Figure~\ref{fig:aprr_up}.

\Subsection{User Participation}
We see that user participation for the majority of all first pull requests, both
merged and not merged, is 0. This indicates that in general, most users are not
attempting to engage in the peripheral activity of commenting on other pull
requests before submitting their own. The GitHub interface makes it relatively
easy for a user to fork a repository, make changes, and submit the changes for
consideration. Previous studies on GitHub have shown that the number of
contributions did increase for some projects that moved from other hosting
options to GitHub~\cite{mcdonald_performance_2013}. It is possible this
interface lowers the barrier of entry for a developer who wants to contribute to
a project, and allows them to bypass participating in the joining script
described by ~\cite{von_krogh_community_2003}.

We also examine these variables for first pull requests by users who later
submit another pull request. Our intuition here is that some users might
encounter a bug they fix or desire a feature that they implement, and then
submit these changes back to repository. They may not comment on other pull
requests as they are not interested in becoming long term members of the
community, but rather are just interested in submitting a one time patch.
Figure~\ref{fig:aprr_up_repeaters} shows a visualization of the same first pull
requests, but only for users who submit at least one other pull request at a
later point in our data set, and Figure~\ref{fig:aprr_up_repeaters_10} shows the
data for users who submit at least 5 more times. Looking at users who submit
at least one other time cuts our number of observations from 13,383 to 5,207,
indicating that approximately 61\% of these pull requests come from users who
will not contribute any others. Looking at users who will submit at least 10
more times gives us a total of 1,155 observations.

It is clear that in all these cases, regardless of whether or not they will be
continuing to submit other pull requests later, at the time of submitting their
first pull request, users are generally not participating in the community. The
previous graphs only consider the number of pull requests a user commented on
before submitting their first pull request, so we do not capture how users who
submit multiple pull request over time comment on other pull requests over time.
In Figure~\ref{fig:commented_pullrequests_totals} we plot the total number of
others' pull requests that a user commented on by how many pull requests they
submitted themselves, considering only users who have submitted at least two
pull requests. There is not a strong correlation between these variables
(Spearman's $\rho$  = 0.44), indicating that users do not necessarily
participate in more commenting as they continue to submit more pull requests.

\begin{figure}[p] \centering \label{fig:aprr_up}
\includegraphics[scale=0.6]{figures/aprr_up_ggplot.png} \caption{User
participation and attention a pull request recieves for all first pull
requests.} \end{figure}

\begin{figure}[p] \centering \label{fig:aprr_up_repeaters}
\includegraphics[scale=0.6]{figures/aprr_up_repeaters_ggplot.png} \caption{User
participation and attention a pull request recieves variables for users who
submit at least one other pull request in our data set.} \end{figure}

\begin{figure}[p] \centering \label{fig:aprr_up_repeaters_10}
\includegraphics[scale=0.6]{figures/aprr_up_repeaters_10_ggplot.png}
\caption{User participation and attention a pull request recieves variables for
users who submit at least 10 other pull requests in our data set.} \end{figure}

\begin{figure}[p] \centering \label{fig:commented_pullrequests_totals}
\includegraphics[scale=0.6]{figures/commented_pullrequests_totals_ggplot.png}
\caption{Total number of pull requests commented on and total number of pull
requests submitted for each user.} \end{figure}

\Subsection{Attention Pull Request Receives}
In Figure~\ref{fig:aprr_up}, we see more variance in the number of comments on
first pull requests than we did with the number of pull requests users commented
on before submitting. However, there is clearly no linear separation of merged
and not merged pull requests using this variable, so it seems just viewing the
amount of activity a pull request receives is not enough to explain whether or
not it gets merged.

To test whether or not the content of these comments is predictive of whether or
not a pull request is merged, we collect the comments for each of our first pull
requests. We ignore comments made by the user who submitted the pull request,
since we are interested in what other users had to say about it. We also ignore
the last comment associated with a pull request, since these often will
explicitly say whether or not the maintainer is merging the pull request or not.
We are more interested in if the type of language used in the discussion of a
pull request is predictive of whether or not it is accepted. We ignore pull
requests that only have one comment associated with it. This gives us a sample
size of 5,674. 3,811, approximately 67\% are unmerged. We treat the remaining
comments associated with the pull request as one document, and convert them into
feature vectors representing the count of each unigram and bigram in the
documents, and train both a logistic regression and naive bayes classifier using
this feature set. The results of testing these classifiers is shown in
Table~\ref{tbl:classifiers}. The results shown are the result of running 10-fold
cross validation. The low recall rates indicate that the text data is not
sufficient to distinguish positive cases. Of course, our sample size of 5,674 is
relatively small, but it is interesting to note that only 42\% of the first pull
requests in our data set have more than 1 comment associated with them.

\begin{table}[ht] \centering \label{tbl:classifiers}
  \caption{Classifier results}
  \begin{tabular}{lll}
  \hline\hline
  ~         & Logistic Regression & Naive Bayes \\
  Accuracy  & 69.6\%              & 70.6\%      \\
  Precision & 56.0\%              & 60.3\%      \\
  Recall    & 36.1\%              & 30.7\%      \\
  \hline
  \end{tabular}
\end{table}



\Section{First Mover Advantage} \label{sec:firstmover}

~\cite{viegas_studying_2004} find what they call a \textit{first-mover
advantage} in the editing of Wikipedia articles, wherein the first contribution
to a page tends to survive longer and recieve less modifications than following
contributions. In this section, we explore how the acceptance of pull requests
changes over time.

In Figure~\ref{fig:acceptance_over_time}, we plot the average number of both
merged and unmerged pull requests over a 12 month period.

\begin{figure}[p] \centering \label{fig:acceptance_over_time}
\includegraphics[scale=0.6]{figures/merged_over_time_excel.png}
\caption{Average number of merged and not merged pull requests over 12 months.}
\end{figure}

\clearpage


%
% APPENDIX
%

% Do the settings of appendices with \appendix command
\appendix

% Then create each appendix using
% \Appendix{title_of_appendix} command
%
% BIBLIOGRAPHY
%
% you have two options: 1) create bibliography manually,
% 2) create bibliography automatically. See BibliographyHelp.pdf file for details.


\bibliographystyle{plain}
\bibliography{Master}

\end{document}  % end of document
